Una vez seleccionados los mejores hiper-parámetros para cada modelo se procedió a probar los mismo clasificando los datos de testing separados al comienzo del trabajo.  Los resultados se muestran en la siguiente tabla. 
\begin{center}
    \begin{tabular}{ | l |  p{8cm} | l |}
    \hline
    \textbf{Modelo} & \textbf{Hiper-parámetros} & \textbf{f0.5} \\ \hline
Decision Tree & Criterio de selección=entropy, altura máxima=14 & 0.68  \\ \hline
 Random Forest & Criterio de selección=entropy, altura máxima=14,cantidad máxima de atributos=sqrt, cantidad de árboles=35 & 0.93 \\ \hline
Gaussian Naive Bayes & N/A & 0.61 \\ \hline
Multinomial Naive Bayes & fit\_prior=True, alpha=1.0 & 0.54 \\ \hline
 Bernoulli Naive Bayes & fit\_prior=True, alpha=1.0 &  0.91 \\ \hline
K Nearest Neighbors & Vecinos considerados = 2, Métrica distancia =  Manhattan &   0.61 \\
    \hline
    \end{tabular}
\end{center}